\subsection{ベクトルヘルムホルツ方程式の導出}

文献Elmerでは$\frac{\partial}{\partial t}=-j\omega$としている。
本書では電磁気学では一般的な$\frac{\partial}{\partial t}=j\omega$とする。

マクスウェル方程式
\begin{align}
\nabla\times\vec{H}&=\vec{i}+j\omega\epsilon\vec{E}\label{fml:maxh}\\
\nabla\times\vec{E}&=-j\omega\mu\vec{H}\label{fml:maxe}
\end{align}

ここでマクスウェルの方程式の解説をしようか。
式(\ref{fml:maxh})は中学校で習ったなんとかの法則,
電流が方位磁針の針を動かすやつ。
式(\ref{fml:maxe})はやっぱり中学校で習った電磁誘導の法則。
$\epsilon$は高校で習った誘電率で,
真空では$8.854\times10^{-12}$[F/m]。
$\mu$は透磁率という値で,
大体の物質では$4\pi\times10^{-7}$(単位忘れた),
鉄などの磁性を持つ物質ではその数千倍以上となる。
また, 方向によって値が異なる場合があり,
一般には場所の関数である$3\times3$行列であらわされるテンソルとなる。
$\vec{i}$は電流減, とか, グダグダ書いているが, 表か箇条書きにすべきだな。

式(\ref{fml:maxh})の両辺に$-j\omega$,
式(\ref{fml:maxe})の両辺に$\mu^{-1}$ をかけて
\begin{align}
\nabla\times\left(-j\omega\vec{H}\right)
&=-j\omega\vec{i}+\omega^2\epsilon\vec{E}\\
\mu^{-1}\left(\nabla\times\vec{E}\right)
&=-j\omega\vec{H}
\end{align}
上の2式において$-j\omega\vec{H}$が共通なので
\begin{align}
\nabla\times\left[
\mu^{-1}\left(\nabla\times\vec{E}\right)
\right]
&=-j\omega\vec{i}+\omega^2\epsilon\vec{E}\\
\nabla\times\left[
\mu^{-1}\left(\nabla\times\vec{E}\right)
\right]-\omega^2\epsilon\vec{E}
&=-j\omega\vec{i}
\end{align}
となる。さらに, 媒質が導電率$\sigma$[S/m]を持つ場合,
$\vec{i}$を$\vec{i}+\sigma\vec{E}$とおいて,
\begin{align}
\nabla\times\left[
\mu^{-1}\left(\nabla\times\vec{E}\right)
\right]-\omega^2\epsilon\vec{E}
&=-j\omega\left(\vec{i}+\sigma\vec{E}\right)\\
&=-j\omega\vec{i}+j\omega\sigma\vec{E}\\
\nabla\times\left[
\mu^{-1}\left(\nabla\times\vec{E}\right)
\right]-\omega^2\epsilon\vec{E}-j\omega\sigma\vec{E}
&=-j\omega\vec{i}\\
\nabla\times\left[
\mu^{-1}\left(\nabla\times\vec{E}\right)
\right]-\omega\left(\omega\epsilon+j\sigma\right)\vec{E}
&=-j\omega\vec{i}
\end{align}
と考えればよい。鉄損を考える場合は$\mu$を複素数とすればよい。

\subsection{弱形式の導出}

基底関数$v$をかけて体積分
文献 Nasa では edge basis function とあるので,
「基底関数」でよいだろう。「辺基底関数」とか言うか?
$N_i$ というか文献 Nasa 内では shape function と書いているが,
これがノードベースFEMの基底関数なんだよな。一気に書くのは大変だから,
分けて書くか。
\begin{align}
\iiint_\Omega\nabla\times\left[
\mu^{-1}\left(\nabla\times\vec{E}\right)
\right]\cdot\vec{v}\,dx\,dy\,dz
\end{align}
ベクトル解析の公式
$\left(\nabla\times\vec{a}\right)\cdot\vec{b}
=\nabla\cdot\left(\vec{a}\times\vec{b}\right)
+\left(\nabla\times\vec{b}\right)\cdot\vec{a}$
に$\vec{a}=\mu^{-1}\left(\nabla\times\vec{E}\right)$, $\vec{b}=\vec{v}$
を適用して
\begin{align}
(\textrm{与式})&=
\iiint_\Omega\nabla\cdot
\left\{\left[\mu^{-1}
\left(\nabla\times\vec{E}\right)
\right]\times\vec{v}\right\}\,dx\,dy\,dz\\
&+\iiint_\Omega\left[\mu^{-1}\left(\nabla\times\vec{E}\right)\right]\cdot
\left(\nabla\times\vec{v}\right)\,dx\,dy\,dz
\end{align}
第1項にガウスの発散定理を適用して
\begin{align}
&\iiint_{\Omega}\nabla\cdot
\left\{\left[\mu^{-1}\left(\nabla\times\vec{E}\right)\right]\times\vec{v}\right\}
\,dx\,dy\,dz\\
=&\iint_\Gamma
\left\{\left[
\mu^{-1}\left(\nabla\times\vec{E}\right)
\right]\times\vec{v}\right\}\cdot\hat{n}
\,d\Gamma
\end{align}
ベクトル解析の公式
$\left(\vec{a}\times\vec{b}\right)\cdot\vec{c}
=\left(\vec{b}\times\vec{c}\right)\cdot\vec{a}$
に
$\vec{a}=\mu^{-1}\left(\nabla\times\vec{E}\right),
 \vec{b}=\vec{v}, \vec{c}=\hat{n}$
を適用して
\begin{align}
\iint_\Gamma
\left(\vec{v}\times\hat{n}\right)\cdot
\left[\mu^{-1}\left(\nabla\times\vec{E}\right)
\right]\,d\Gamma
\end{align}

\begin{align}
\nabla\times\vec{E}=-j\omega\mu\vec{H}
\end{align}

境界で無反射とすると
\begin{align}
\hat{n}\times\vec{E}=&\sqrt\frac{\mu}{\epsilon}\vec{H}\\
-j\omega\sqrt{\epsilon\mu}\left(\hat{n}\times\vec{E}\right)=&-j\omega\mu\vec{H}
\end{align}

境界面積分項は
\begin{align}
&-j\omega\iint_\Gamma
\left(\vec{v}\times\hat{n}\right)\cdot
\left[\mu^{-1}
\sqrt{\epsilon\mu}\left(\hat{n}\times\vec{E}\right)
\right]\,d\Gamma\\
=&j\omega\iint_\Gamma
\left(\hat{n}\times\vec{v}\right)\cdot
\left[\mu^{-1}
\sqrt{\epsilon\mu}\left(\hat{n}\times\vec{E}\right)
\right]\,d\Gamma
\end{align}
となる。ああ、どのタイミングで
$\epsilon$, $\mu$をテンソルからスカラにしようか。