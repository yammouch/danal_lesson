\subsection{ベクトルヘルムホルツ方程式の導出}

文献Elmerでは$\frac{\partial}{\partial t}=-j\omega$としている。
本書では電磁気学では一般的な$\frac{\partial}{\partial t}=j\omega$とする。

マクスウェル方程式
\begin{align}
\nabla\times\vec{H}&=\vec{i}+j\omega\epsilon\vec{E}\label{fml:maxh}\\
\nabla\times\vec{E}&=-j\omega\mu\vec{H}\label{fml:maxe}
\end{align}

ここでマクスウェルの方程式の解説をしようか。
式(\ref{fml:maxh})は中学校で習ったなんとかの法則,
電流が方位磁針の針を動かすやつ。
式(\ref{fml:maxe})はやっぱり中学校で習った電磁誘導の法則。
$\epsilon$は高校で習った誘電率で,
真空では$8.854\times10^{-12}$[F/m]。
$\mu$は透磁率という値で,
大体の物質では$4\pi\times10^{-7}$(単位忘れた),
鉄などの磁性を持つ物質ではその数千倍以上となる。
また, 方向によって値が異なる場合があり,
一般には場所の関数である$3\times3$行列であらわされるテンソルとなる。
$\vec{i}$は電流減, とか, グダグダ書いているが, 表か箇条書きにすべきだな。

式(\ref{fml:maxh})の両辺に$-j\omega$,
式(\ref{fml:maxe})の両辺に$\mu^{-1}$ をかけて
\begin{align}
\nabla\times\left(-j\omega\vec{H}\right)
&=-j\omega\vec{i}+\omega^2\epsilon\vec{E}\\
\mu^{-1}\left(\nabla\times\vec{E}\right)
&=-j\omega\vec{H}
\end{align}
上の2式において$-j\omega\vec{H}$が共通なので
\begin{align}
\nabla\times\left[
\mu^{-1}\left(\nabla\times\vec{E}\right)
\right]
&=-j\omega\vec{i}+\omega^2\epsilon\vec{E}\\
\nabla\times\left[
\mu^{-1}\left(\nabla\times\vec{E}\right)
\right]-\omega^2\epsilon\vec{E}
&=-j\omega\vec{i}
\end{align}
となる。さらに, 媒質が導電率$\sigma$[S/m]を持つ場合,
$\vec{i}$を$\vec{i}+\sigma\vec{E}$とおいて,
\begin{align}
\nabla\times\left[
\mu^{-1}\left(\nabla\times\vec{E}\right)
\right]-\omega^2\epsilon\vec{E}
&=-j\omega\left(\vec{i}+\sigma\vec{E}\right)\\
&=-j\omega\vec{i}+j\omega\sigma\vec{E}\\
\nabla\times\left[
\mu^{-1}\left(\nabla\times\vec{E}\right)
\right]-\omega^2\epsilon\vec{E}-j\omega\sigma\vec{E}
&=-j\omega\vec{i}\\
\nabla\times\left[
\mu^{-1}\left(\nabla\times\vec{E}\right)
\right]-\omega\left(\omega\epsilon+j\sigma\right)\vec{E}
&=-j\omega\vec{i}
\end{align}
と考えればよい。鉄損を考える場合は$\mu$を複素数とすればよい。

\subsection{弱形式の導出}

