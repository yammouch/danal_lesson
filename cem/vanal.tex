\subsection{ベクトル解析の公式}

\begin{align}
&\nabla\times\left(\nabla f\right)\\
=&\epsilon_{ijk}\partial_j\partial_k f
\end{align}
$i=0$を考えて
\begin{align}
&\epsilon_{0jk}\partial_j\partial_k f\\
=&\partial_1\partial_2 f-\partial_2\partial_1 f\\
=&0
\end{align}
$i=1$, 2の場合も同様に0になる。したがって
\begin{align}
\nabla\times\left(\nabla f\right)=\bm{0}
\end{align}

\begin{align}
&\nabla\cdot\left(\vec{a}\times\vec{b}\right)\\
=&\epsilon_{ijk}\partial_i\left(a_jb_k\right)\\
=&b_k\epsilon_{ijk}\left(\partial_ia_j\right)
 +a_j\epsilon_{ijk}\left(\partial_ib_k\right)\\
=&b_k\epsilon_{kij}\left(\partial_ia_j\right)
 -a_j\epsilon_{jik}\left(\partial_ib_k\right)\\
=&\vec{b}\cdot\left(\nabla\times\vec{a}\right)
 -\vec{a}\cdot\left(\nabla\times\vec{b}\right)
\end{align}

\begin{align}
&\vec{a}\cdot\left(\vec{b}\times\vec{c}\right)\\
=&a_i\epsilon_{ijk}b_jc_k\\
=&\epsilon_{ijk}a_ib_jc_k\\
=&b_j\epsilon_{ijk}c_ka_i=b_j\epsilon_{jki}c_ka_i
 =\vec{b}\cdot\left(\vec{c}\times\vec{a}\right)\\
=&c_k\epsilon_{ijk}a_ib_j=c_k\epsilon_{kij}a_ib_j
 =\vec{c}\cdot\left(\vec{a}\times\vec{b}\right)
\end{align}

\begin{align}
&\nabla\times\left(f\vec{a}\right)\\
=&\epsilon_{ijk}\partial_j\left(fa_k\right)\\
=&\epsilon_{ijk}a_k\partial_jf+f\epsilon_{ijk}\partial_ja_k\\
=&\left(\nabla f\right)\times\vec{a}+f\left(\nabla\times\vec{a}\right)
\end{align}

\begin{align}
&\vec{a}\times\left(\vec{b}\times\vec{c}\right)\\
=&\epsilon_{ijk}a_j\epsilon_{klm}b_lc_m
\end{align}
$\left(i,j,k\right)=\left(0,1,2\right)$
を考えて
\begin{align}
&a_1\left(\epsilon_{2lm}b_lc_m\right)\\
=&a_1\left(b_0c_1-b_1c_0\right)\\
=&a_1b_0c_1-a_1b_1c_0
\end{align}
同様
$\left(i,j,k\right)=\left(0,2,1\right)$
を考えて
\begin{align}
&-a_2\left(\epsilon_{1lm}b_lc_m\right)\\
=&-a_2\left(b_2c_0-b_0c_2\right)\\
=&-a_2b_2c_0+a_2b_0c_2
\end{align}
和をとって
\begin{align}
&a_1b_0c_1+a_2b_0c_2-a_1b_1c_0-a_2b_2c_0\\
=&b_0\left(a_1c_1+a_2c_2\right)-c_0\left(a_1b_1+a_2b_2\right)\\
=&b_0\left(a_0c_0+a_1c_1+a_2c_2\right)-c_0\left(a_0b_0+a_1b_1+a_2b_2\right)\\
=&b_0\left(\bm{a}\cdot\bm{c}\right)-c_0\left(\bm{a}\cdot\bm{b}\right)
\end{align}
同様に$i=1$, 2の場合はそれぞれ
\begin{align}
b_1\left(\bm{a}\cdot\bm{c}\right)
-c_1\left(\bm{a}\cdot\bm{b}\right)\\
b_2\left(\bm{a}\cdot\bm{c}\right)
-c_2\left(\bm{a}\cdot\bm{b}\right)
\end{align}
となる。まとめると
\begin{align}
\bm{a}\times\left(\bm{b}\times\bm{c}\right)
=&\left(\begin{array}{c}b_0\\b_1\\b_2\end{array}\right)\left(\bm{a}\cdot\bm{c}\right)
-\left(\begin{array}{c}c_0\\c_1\\c_2\end{array}\right)\left(\bm{a}\cdot\bm{b}\right)\\
=&\bm{b}\left(\bm{a}\cdot\bm{c}\right)
-\bm{c}\left(\bm{a}\cdot\bm{b}\right)
\end{align}
となる。

\subsection{積分公式}

いくつかある積分公式もAppendixに追いやるべきなような気がしてきた。

\begin{align}
&\int_a^b\left(x-a\right)^i\left(b-x\right)^j\,dx\\
=&\frac{\left[\left(x-a\right)^{i+1}\left(b-x\right)^j\right]_{x=a}^{b}}{i+1}
 - \frac{j}{i+1}\int_a^b-\left(x-a\right)^{i+1}\left(b-x\right)^{j-1}\,dx\\
=&\frac{j}{i+1}\int_a^b\left(x-a\right)^{i+1}\left(b-x\right)^{j-1}\,dx\\
=&\frac{j\left(j-1\right)}{\left(i+1\right)\left(i+2\right)}
  \int_a^b\left(x-a\right)^{i+2}\left(b-x\right)^{j-2}\,dx\\
=&\frac{j\left(j-1\right)\ldots2\cdot1}
       {\left(i+1\right)\left(i+2\right)
        \ldots\left(i+j-1\right)\left(i+j\right)}
  \int_a^b\left(x-a\right)^{i+j}\,dx\\
=&\frac{i!j!}{\left(i+j\right)!}
  \frac{\left[\left(x-a\right)^{i+j+1}\right]_{x=a}^b}
       {i+j+1}\\
=&\frac{i!j!}{\left(i+j+1\right)!}\left(b-a\right)^{i+j+1}
\end{align}
実は1/6公式の一般化なんだね。第一種オイラー積分というらしい。

\begin{align}
&\int_0^1\int_0^{1-N_1}\int_0^{1-N_1-N_2}
{N_1}^k{N_2}^l{N_3}^m
\,dN_3\,dN_2\,dN_1\\
=&
\int_0^1{N_1}^k\int_0^{1-N_1}{N_2}^l\int_0^{1-N_1-N_2}{N_3}^m
\,dN_3\,dN_2\,dN_1\\
=&
\frac{1}{m+1}
\int_0^1{N_1}^k\int_0^{1-N_1}
\left[{N_3}^{m+1}\right]_{N_3=0}^{1-N_1-N_2}
\,dN_2\,dN_1\\
=&\frac{1}{m+1}
\int_0^1\int_0^{1-N_1}
{N_2}^l\left(1-N_1-N_2\right)^{m+1}
\,dN_2\,dN_1\\
\end{align}
公式
$\displaystyle\int_a^b\left(x-a\right)^i\left(b-x\right)^j\,dx
=\frac{i!j!}{\left(i+j+1\right)!}\left(b-a\right)^{i+j+1}$
に
$a=0$, $b=1-N_1$, $x=N_2$, $i=l$, $j=m+1$
を適用して
\begin{align}
&\frac{1}{m+1}\frac{l!\left(m+1\right)!}{\left(l+m+2\right)!}
\int_0^1{N_1}^k
\left(1-N_1\right)^{l+m+2}
\,dN_1\\
=&\frac{l!m!}{\left(l+m+2\right)!}
\int_0^1{N_1}^k
\left(1-N_1\right)^{l+m+2}
\,dN_1
\end{align}
公式
$\displaystyle\int_a^b\left(x-a\right)^i\left(b-x\right)^j\,dx
=\frac{i!j!}{\left(i+j+1\right)!}\left(b-a\right)^{i+j+1}$
に
$a=0$, $b=1$, $x=N_1$, $i=k$, $j=l+m+2$
を適用して
\begin{align}
&\frac{l!m!}{\left(l+m+2\right)!}
\frac{k!\left(l+m+2\right)!}
     {\left(k+l+m+3\right)}\\
=&\frac{k!l!m!}{\left(k+l+m+3\right)!}
\end{align}
となる。

次。$\bm{a}\cdot\nabla$をどこに書くかは迷いどころ。
\begin{align}
&\nabla\left(\bm{a}\cdot\bm{b}\right)\\
=&\partial_i\left(a_jb_j\right)\\
=&b_j\partial_ia_j+a_j\partial_ib_j\\
=&
\left[
\left(
\begin{array}{ccc}b_0 & b_1 & b_2\end{array}
\right)
\left(
\begin{array}{c}
\partial_0\\\partial_1\\\partial_2
\end{array}
\right)
\left(
\begin{array}{ccc}a_0 & a_1 & a_2\end{array}
\right)
\right]^T\\
+&\left[
\left(
\begin{array}{ccc}a_0 & a_1 & a_2\end{array}
\right)
\left(
\begin{array}{c}
\partial_0\\\partial_1\\\partial_2
\end{array}
\right)
\left(
\begin{array}{ccc}b_0 & b_1 & b_2\end{array}
\right)
\right]^T\\
=&\left(\bm{b}^T\nabla\bm{a}^T\right)^T
 +\left(\bm{a}^T\nabla\bm{b}^T\right)^T\\
=&\left(\bm{b}\cdot\nabla\right)\bm{a}
 +\left(\bm{a}\cdot\nabla\right)\bm{b}
\end{align}

\begin{align}
\left(\bm{a}\cdot\nabla\right)\bm{b}
&=a_j\nabla b_j\\
&=a_j\partial_i b_j\\
&=
\left(
\begin{array}{c}
a_0\partial_0 b_0 +
a_1\partial_0 b_1 +
a_2\partial_0 b_2 \\
a_0\partial_1 b_0 +
a_1\partial_1 b_1 +
a_2\partial_1 b_2 \\
a_0\partial_1 b_0 +
a_1\partial_1 b_1 +
a_2\partial_1 b_2
\end{array}
\right)\\
&=
\left(
\begin{array}{ccc}
\partial_0 b_0 &
\partial_0 b_1 &
\partial_0 b_2 \\
\partial_1 b_0 &
\partial_1 b_1 &
\partial_1 b_2 \\
\partial_2 b_0 &
\partial_2 b_1 &
\partial_2 b_2
\end{array}
\right)
\left(
\begin{array}{c}
a_0\\a_1\\a_2\end{array}
\right)\\
&=
\left[
\left(
\begin{array}{ccc}a_0 & a_1 & a_2\end{array}
\right)
\left(
\begin{array}{ccc}
\partial_0 b_0 &
\partial_1 b_0 &
\partial_2 b_0 \\
\partial_0 b_1 &
\partial_1 b_1 &
\partial_2 b_1 \\
\partial_0 b_2 &
\partial_1 b_2 &
\partial_2 b_2
\end{array}
\right)
\right]^T\\
&=
\left[
\left(
\begin{array}{ccc}a_0 & a_1 & a_2\end{array}
\right)
\left(
\begin{array}{c}
\partial_0 \\
\partial_1 \\
\partial_2
\end{array}
\right)
\left(
\begin{array}{ccc}b_0 & b_1 & b_2\end{array}
\right)
\right]^T\\
&=\left(\bm{a}^T\nabla\bm{b}^T\right)^T
\end{align}

こーれーだー! 必要あらばこんな感じに色々書くことができる,
として引用しよう。

%\begin{align}
%\left(\bm{a}\cdot\nabla\right)\bm{b}
%&=a_j\nabla b_j\\
%&=a_j\partial_i b_j\\
%&=
%\left[
%\left(
%\begin{array}{ccc}
%a_0 & a_1 & a_2
%\end{array}
%\right)
%\left(
%\begin{array}{c}
%\partial_i b_0\\
%\partial_i b_1\\
%\partial_i b_2
%\end{array}
%\right)
%\right]^T\\
%&=
%\left[
%\left(
%\begin{array}{ccc}
%a_0 & a_1 & a_2
%\end{array}
%\right)
%\left(
%\begin{array}{ccc}
%\partial_0 b_0 &
%\partial_1 b_0 &
%\partial_2 b_0 \\
%\partial_0 b_1 &
%\partial_1 b_1 &
%\partial_2 b_1 \\
%\partial_0 b_2 &
%\partial_1 b_2 &
%\partial_2 b_2
%\end{array}
%\right)
%\right]^T
%\end{align}

%いまいちだな。$a_j\partial_i b_j$を一度展開したほうがよさそう。

%なんか違う。
%いや, 合っている。
%\begin{align}
%b_j\partial_ia_j=
%\left[
%\left(
%\begin{array}{c}
%\partial_i a_0\\\partial_i a_1\\\partial_i a_2
%\end{array}
%\right)
%\left(
%\begin{array}{ccc}
%b_0 & b_1 & b_2\end{array}
%\right)
%\right]^T\\
% +\left(\bm{a}\cdot\nabla\right)\bm{b}
%\end{align}