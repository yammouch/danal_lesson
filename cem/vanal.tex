\subsection{ベクトル解析の公式}

%\begin{align}
%&\vec{a}\cdot\left(\nabla\times\vec{b})\\
%=&a_i\epsilon_{ijk}\partial_jb_k\\
%=&
%\end{align}

\begin{align}
&\nabla\cdot\left(\vec{a}\times\vec{b}\right)\\
=&\epsilon_{ijk}\partial_i\left(a_jb_k\right)\\
=&b_k\epsilon_{ijk}\left(\partial_ia_j\right)
 +a_j\epsilon_{ijk}\left(\partial_ib_k\right)\\
=&b_k\epsilon_{kij}\left(\partial_ia_j\right)
 -a_j\epsilon_{jik}\left(\partial_ib_k\right)\\
=&\vec{b}\cdot\left(\nabla\times\vec{a}\right)
 -\vec{a}\cdot\left(\nabla\times\vec{b}\right)
\end{align}

\begin{align}
&\vec{a}\cdot\left(\vec{b}\times\vec{c}\right)\\
=&a_i\epsilon_{ijk}b_jc_k\\
=&\epsilon_{ijk}a_ib_jc_k\\
=&b_j\epsilon_{ijk}c_ka_i=b_j\epsilon_{jki}c_ka_i
 =\vec{b}\cdot\left(\vec{c}\times\vec{a}\right)\\
=&c_k\epsilon_{ijk}a_ib_j=c_k\epsilon_{kij}a_ib_j
 =\vec{c}\cdot\left(\vec{a}\times\vec{b}\right)
\end{align}

\begin{align}
&\nabla\times\left(f\vec{a}\right)\\
=&\epsilon_{ijk}\partial_j\left(fa_k\right)\\
=&\epsilon_{ijk}a_k\partial_jf+f\epsilon_{ijk}\partial_ja_k\\
=&\left(\nabla f\right)\times\vec{a}+f\left(\nabla\times\vec{a}\right)
\end{align}

\subsection{積分公式}

いくつかある積分公式もAppendixに追いやるべきなような気がしてきた。

\begin{align}
&\int_a^b\left(x-a\right)^i\left(b-x\right)^j\,dx\\
=&\frac{\left[\left(x-a\right)^{i+1}\left(b-x\right)^j\right]_{x=a}^{b}}{i+1}
 - \frac{j}{i+1}\int_a^b-\left(x-a\right)^{i+1}\left(b-x\right)^{j-1}\,dx\\
=&\frac{j}{i+1}\int_a^b\left(x-a\right)^{i+1}\left(b-x\right)^{j-1}\,dx\\
=&\frac{j\left(j-1\right)}{\left(i+1\right)\left(i+2\right)}
  \int_a^b\left(x-a\right)^{i+2}\left(b-x\right)^{j-2}\,dx\\
=&\frac{j\left(j-1\right)\ldots2\cdot1}
       {\left(i+1\right)\left(i+2\right)
        \ldots\left(i+j-1\right)\left(i+j\right)}
  \int_a^b\left(x-a\right)^{i+j}\,dx\\
=&\frac{i!j!}{\left(i+j\right)!}
  \frac{\left[\left(x-a\right)^{i+j+1}\right]_{x=a}^b}
       {i+j+1}\\
=&\frac{i!j!}{\left(i+j+1\right)!}\left(b-a\right)^{i+j+1}
\end{align}
実は1/6公式の一般化なんだね。

\begin{align}
&\int_0^1\int_0^{1-N_1}\int_0^{1-N_1-N_2}
{N_1}^k{N_2}^l{N_3}^m
\,dN_3\,dN_2\,dN_1\\
=&
\int_0^1{N_1}^k\int_0^{1-N_1}{N_2}^l\int_0^{1-N_1-N_2}{N_3}^m
\,dN_3\,dN_2\,dN_1\\
=&
\frac{1}{m+1}
\int_0^1{N_1}^k\int_0^{1-N_1}
\left[{N_3}^{m+1}\right]_{N_3=0}^{1-N_1-N_2}
\,dN_2\,dN_1\\
=&\frac{1}{m+1}
\int_0^1\int_0^{1-N_1}
{N_2}^l\left(1-N_1-N_2\right)^{m+1}
\,dN_2\,dN_1\\
\end{align}
公式
$\displaystyle\int_a^b\left(x-a\right)^i\left(b-x\right)^j\,dx
=\frac{i!j!}{\left(i+j+1\right)!}\left(b-a\right)^{i+j+1}$
に
$a=0$, $b=1-N_1$, $x=N_2$, $i=l$, $j=m+1$
を適用して
\begin{align}
&\frac{1}{m+1}\frac{l!\left(m+1\right)!}{\left(l+m+2\right)!}
\int_0^1{N_1}^k
\left(1-N_1\right)^{l+m+2}
\,dN_1\\
=&\frac{l!m!}{\left(l+m+2\right)!}
\int_0^1{N_1}^k
\left(1-N_1\right)^{l+m+2}
\,dN_1
\end{align}
公式
$\displaystyle\int_a^b\left(x-a\right)^i\left(b-x\right)^j\,dx
=\frac{i!j!}{\left(i+j+1\right)!}\left(b-a\right)^{i+j+1}$
に
$a=0$, $b=1$, $x=N_1$, $i=k$, $j=l+m+2$
を適用して
\begin{align}
&\frac{l!m!}{\left(l+m+2\right)!}
\frac{k!\left(l+m+2\right)!}
     {\left(k+l+m+3\right)}\\
=&\frac{k!l!m!}{\left(k+l+m+3\right)!}
\end{align}
となる。