\subsection{複数要素のシミュレーション}

本節では1要素のシミュレーションを複数要素に拡張します。解くべき方程式は
(数式)
です。右辺は1要素シミュレーションと同様の電流励起です。左辺においては
\begin{align}
R_k,g(l),g(m) &= r_k,l,m\\
S_k,g(l),g(m) &= s_k,l,m
\end{align}
です。$R_k, r_k, S_k, s_k$ とその添字(そえじ)の関係を図?に示します。
$R_k, S_k$ はエッジ数の大きさを持つ正方行列です。
$r_k, s_k$ は大きさ 4 の正方行列で,
前節の1メッシュシミュレーションでの$r, s$に相当します。
1番目の添え字(ここでは$k$)はメッシュ番号を表します。
2番目の添え字(ここでは$g(l), l$)は行列
$R_k, r_k, S_k, s_k$ の行番号を表します。
3番目の添え字(ここでは$g(m), m$)は行列
$R_k, r_k, S_k, s_k$ の列番号を表します。
$g$はローカルエッジ番号をグローバルエッジ番号に変換する関数です。
各メッシュ毎に $r_k, s_k$ を計算しそれに係数を掛けグローバルな
$R_k, S_k$ に加算すればよいことになります。
これを行うコードをリスト?に示します。
この関数を使用して図?に示す例題を1つずつシミュレーションしていきます。
